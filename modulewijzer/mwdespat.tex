\documentclass[titlepage,a4paper, 11pt]{article}

\usepackage[dutch]{babel}
\usepackage{graphicx,fancyhdr,hyperref,a4wide}	

%\setlength{\parindent}{0.0in}     % inspringen bij nieuwe alinea?
%\setlength{\parskip}{0.1in}       %ruimte laten tussen alinea' s?
%\newcommand{\HRule}{\rule{\linewidth}{0.5mm}}

\newcommand{\mcode}{TIRDES01}  %modulecode
\newcommand{\mname}{Design patterns} %modulenaam
\newcommand{\HRule}{\rule{\linewidth}{1pt}} % dikke horizontale lijn voor frontpagina

\renewcommand{\headrulewidth}{0.4pt}     %dikte van headerlijn
\renewcommand{\footrulewidth}{0.4pt}     %dikte van footerlijn

\headheight = 70pt                        %body van pagina verder naar beneden
\voffset = -40pt                          %corrigeren, anders valt footer van de pagina
\lhead{\large Modulewijzer}               %linker headertekst
\chead{\large Hogeschool Rotterdam}       %centraal headertekst
\rhead{\includegraphics[width=2cm]{logo}} %rechter headertekst met logo

\lfoot{\mcode\ \today}                    %linker foot tekst: modulecode en datum
\cfoot{}                                  %center foot tekst leeg

\pagestyle{fancy}                         % het fancyheaders pakket van van Oostrum gebruiken


\begin{document}
\sffamily
%%%%%%%%%%%%%%%%%%%%%%%%%%%%FRONTPAGINA%%%%%%%%%%%%%%%%%%%
\begin{titlepage}

\thispagestyle{fancy}
\ \\
\ \\
\ \\
\begin{center}

% Titel
  \textsc{\LARGE Hogeschool Rotterdam / CMI}\\[1.5cm]

  \HRule \\[0.4cm]
  { \Huge \bfseries \mname}\\[0.4cm]
  \ \\
  \ \\
  { \large \bfseries \mcode}\\[0.4cm]

  \HRule \\[1.5cm]

  \vfill
  % Author and supervisor
  \begin{minipage}{0.5\textwidth}
    \begin{flushleft}
      Aantal studieunten: 3 ects\\
      Modulebeheerder: Wessel Oele
    \end{flushleft}
\end{minipage}
\begin{minipage}{0.4\textwidth}
  \begin{flushright}
		\begin{tabular}{ | l l |}
		  \hline
		  Goedgekeurd door: &\ \\
		  \textbf{(namens toetscommissie)} & \ \\
		  Datum: & \ \\
		  \hline
		\end{tabular}
  \end{flushright}
\end{minipage}
\end{center}
\end{titlepage}
%%%%%%%%%%%%%%%%%%%%%%%%%%%%FRONTPAGINA%%%%%%%%%%%%%%%%%%%
\rfoot{\thepage} %pagina nummering vanaf inhoudsopgave
\tableofcontents %inhoudsopgave
\newpage
%%%%%%%%%%%%%%%%%%%%%%%%%%%%MODULE A4-TJE%%%%%%%%%%%%%%%%%
\section*{Modulebeschrijving}
\scriptsize
\begin{tabular}{|p{3cm}|p{11cm}|}
\hline
Modulenaam:&\mname\\
\hline
Modulecode:&\mcode\\
\hline
Aantal studiepunten en studiebelastingsuren:&Deze module levert 3 studiepunten op.
\begin{itemize}
\item 8 $\times$ 120 minuten hoorcollege
\item 8 $\times$ 120 minuten practicum
\item 12 $\times$ 120 minuten zelfstudie
\end{itemize}
\\
\hline
Vereiste voorkennis:&Java: basis, o.o.p., toepassen, datastructuren\\
\hline
Werkvorm:&hoorcollege en practicum\\
\hline
Toetsing:&Practicumopdrachten\\
\hline
Leermiddelen:&Design patterns explained, auteur: Alan Shalloway, James R. Trott, uitgever: Addison Wesley, ISBN: 978-0-321-24714-8\\
\hline
Draagt bij aan competentie&\begin{center}
\includegraphics[width=7cm]{comptabel}
\end{center}\\
\hline
Leerdoelen:&Kunnen werken met U.M.L. Begrijpen en kunnen toepassen van diverse design patterns zoals adapter, facade, bridge, abstract factory, e.v.a.\\
\hline
Inhoud:&
\begin{itemize}
\item U.M.L.
\item Diverse design patterns
\end{itemize}\\
\hline
Opmerkingen:&\\
\hline
Modulebeheerder:&Wessel Oele\\
\hline 
Datum:&\today\\
\hline
\end{tabular}
%%%%%%%%%%%%%%%%%%%%%%%%%%%%MODULE A4-TJE%%%%%%%%%%%%%%%%%
\newpage
\normalsize
\section{Algemene omschrijving}
Het correct ontwerpen van software is een gecompliceerde zaak. Niet alleen is het ontwerpen zelf niet eenvoudig, moeilijker wordt het wanneer de eisen, waaraan een stuk software moet voldoen ook nog eens veranderen. Ten slotte zijn de meeste computerprogramma' s nooit af en dient er in de jaren na oplevering met enige regelmaat aan onderhoud en uitbreiding gedaan te worden. 

Design patterns vormen een hulpmiddel bij het ontwerpen van software, opdat software zo ontworpen kan worden dat het aanpassen en uitbreiden van de software eenvoudiger wordt. Ook vergroot het gebruik van design patterns de herbruikbaarheid van (delen van) de software.

Design patterns vinden hun oorsprong in het einde van de jaren '70 toen Christopher Alexander patronen als hulpmiddel gebruikte bij het ontwerpen van gebouwen en steden. Na enig experimenteel werk van Beck en Cunningham werden design patterns populair in de jaren '90 toen de zogeheten ``gang of four'' (Erich Gamma, Richard Helm, Ralph Johnson en John Vlissides) het boek \emph{``Design Patterns: Elements of Reusable Object-Oriented Software''} publiceerden.
\subsection{Relatie met andere onderwijseenheden}
Deze module bouwt voort op de modules tinpro01-1, tinpro01-2, tinpro01-3, en tinpro01-4. Verondersteld wordt het programmeren in een imperatieve en objectgeori\"enteerde taal te beheersen. 
\subsection{Leermiddelen}
Verplicht:
\begin{itemize}
\item Boek: Design patterns explained, auteur: Alan Shalloway, James R. Trott, uitgever: Addison Wesley, ISBN: 978-0-321-24714-8
\item Software: Java Development Kit (JDK) versie 6, te downloaden van \url{http://www.javasoft.com}
\item Presentaties die gebruikt worden in de hoorcolleges (pdf): te vinden op \url{http://med.hro.nl/oelew}
\item Opdrachten, waaraan gewerkt wordt tijdens het practicum (pdf): te vinden op \url{http://med.hro.nl/oelew}
\end{itemize}
Facultatief:
\begin{itemize}
\item Boek: Design Patterns: Elements of Reusable Object-Oriented Software, auteur: Gamma, Helm, Johnson, Vlissides ,uitgever: Addison-Wesley. ISBN 0-201-63361-2
\item Text editors: Emacs, VI, Jedit, Gedit, etc.
\end{itemize}

\section{Programma}

\begin{tabular}{|p{1cm}|p{4cm}|p{4cm}|p{4cm}|}
\hline
Week&Literatuur&Lesinhoud&Producten\\
\hline
1&D.P. Explained t/m blz. 45,&introductie, herhaling o.o.p., u.m.l.&\\
\hline
2&D.P. Explained t/m blz. 73&o.o.p. problemen&inleveren groepsopdracht 1\\
\hline
3&D.P. Explained t/m blz. 115&facade en adapter pattern&\\
\hline
4&D.P. Explained t/m blz. 136&denkwijze en perspectief&inleveren groepsopdracht 2\\
\hline
5&D.P. Explained t/m blz. 157&strategy pattern&\\
\hline
6&D.P. Explained t/m blz. 191&bridge pattern&inleveren groepsopdracht 3\\
\hline
7&D.P. Explained t/m blz. 212&abstract factory pattern&\\
\hline
8&D.P. Explained t/m blz. 266 &&inleveren groepsopdracht 4\\
\hline
9&vragenuur/inhaalles&&\\
\hline
10&&vragenuur/inhaalles&\\
\hline
\end{tabular}
\section{Toetsing en beoordeling}
\subsection{Procedure}
Deze module wordt getoetst middels 4 groepsopdrachten. Voorwaarden:
\begin{itemize}
\item Opdrachten worden op papier en tijdens de practicumlessen ingeleverd.
\item Een groep bestaat uit maximaal vier studenten. Deze leveren gezamenlijk \'e\'en opdracht in.
\item Bij het inleveren zijn alle leden van de groep aanwezig, opdat \emph{elk} lid de uitwerking mondeling kan verdedigen.
\item De practicumdocent kan naar eigen inzicht (bijvoorbeeld om didactische redenen) afwijken van de practicumopgaven en alternatieve opdrachten aanbieden. Hierbij staat duidelijkheid en integriteit richting de studenten uiteraard voorop. 
\end{itemize}
\newpage
\section{Bijlage 1: Toetsmatrijs}
\begin{tabular}{|p{1cm}|p{4cm}|p{4cm}|p{4cm}|}
\hline
&Leerdoelen&Dublin descriptoren&Verwijzing naar opdracht / vraag / criteria\\
\hline
1&o.o.p., u.m.l.&1,2,3,4&practicumopdracht 1 t/m 4\\
\hline
2&adapter, facade&1,2,3,4&practicumopdracht 2 t/m 4\\
\hline
3&bridge&1,2,3,4&practicumopdracht 3 t/m 4\\
\hline
4&abstract factory&1,2,3,4&practicumopdracht 4\\
\hline
\end{tabular}\\
\vspace{1cm}\\
Dublin-descriptoren:
\begin{enumerate}
\item Kennis en inzicht
\item Toepassen kennis en inzicht
\item Oordeelsvorming
\item Communicatie
\end{enumerate}
\newpage

\section{Bijlage 3: Studielast (normering in ecs)}
\begin{tabular}{|l|p{3cm}|p{3cm}|p{2cm}|}
\hline
&aantal weken&aantal lesuren van 50 minuten&klokuren\\
\hline
\emph{lesuren}&10&4&33\\
\hline
&&&\\
\hline
\emph{zelfstudie}&&&\\
\hline
&&&\\
\hline
leestijd&aantal pagina' s&&\\
\hline
&& 3 per uur&\\
\hline
&& 6 per uur&\\
\hline
&120& 10 per uur&12\\
\hline
&&&\\
\hline
presentaties&&&\\
\hline
&&&\\
\hline
overlegtijd&&&\\
\hline
&&&\\
\hline
uitzoektijd/research&&&11\\
\hline
&&&\\
\hline
niet ingeroosterde lestijd&&&\\
\hline
&&&\\
\hline
\emph{toetsen}&voorbereiden&&3.5\\
\hline
&toets&&1.5\\
\hline
&nabespreking&&1\\
\hline
&&&\\
\hline
\emph{werkstuk,verslag,rapport,scriptie}&uitzoeken&&\\
\hline
&overleggen&&\\
\hline
&schrijven&&\\
\hline
&&&\\
\hline
Stage, Praktijkopdracht&voorbereiding&&\\
\hline
&aanwezigheid&&\\
\hline
&overleg&&\\
\hline
&&&\\
\hline
\emph{Subtotaal in klokuren}&&&56\\
\hline
\emph{Ruis 5\%}&&&\\
\hline
\emph{Totaal in klokuren}&&&56\\
\hline
\emph{Totaal in studiepunten (ects)}&&&2\\
\hline
\end{tabular}
\end{document}



\documentclass{beamer}

\mode<presentation>

\usepackage[dutch]{babel}
%\usepackage{beamerthemesplit}
\usepackage{hyperref}

\usetheme{Berlin}
\useinnertheme{rounded}
\usecolortheme{rose}
\setbeamertemplate{navigation symbols}{} 
\title{Design patterns}
\author{W. Oele}
\date{\today}

\begin{document}
\frame{\titlepage}

\section{Design patterns}

\begin{frame}
\frametitle{Deze les}
\begin{itemize}
\item introductie design patterns
\end{itemize}

\end{frame}


\begin{frame}
\frametitle{Design patterns}
Vragen van begin jaren '90:
\begin{itemize}
\item Komen problemen bij het maken van software iedere keer weer terug in een net iets andere vorm?
\item Is het handig deze veel voorkomende problemen onder te brengen in een verzameling patronen?
\item Kun je bij het oplossen van een probleem de oplossing makkelijker vinden als je een bestaand patroon herkent in het probleem?
\end{itemize}
\end{frame}

\begin{frame} \frametitle{Categorisatie}

\begin{itemize}
 \item  Design patterns zijn het resultaat van een stelselmatige categorisatie van veel voorkomende, zich iedere keer herhalende, problemen bij het ontwerpen en aanpassen van software.\pause
 \item Het is niet gezegd dat je middels categorisatie tot een beter inzicht in de materie komt of dat dit leidt tot vernieuwingen.\pause
 \item Categorisatie is echter een stelselmatige manier om overzicht te cree\"eren en \emph{kan} dus zijn nut hebben\ldots Beter dan niets. \pause
\end{itemize}
\end{frame}

\begin{frame} \frametitle{Wat is een design pattern?}
  Een design pattern bestaat uit een aantal elementen:
  \begin{itemize}
  \item de naam van het patroon
  \item het doel van het patroon: Wat los je ermee op?
  \item hoe bereik je die oplossing?
  \item wat zijn de gevolgen en waarmee rekening te houden?
  \end{itemize}
\end{frame}

\begin{frame} \frametitle{Waarom design patterns te gebruiken?}
  Twee belangrijke redenen:
  \begin{itemize}
  \item hergebruik van eerder gevonden oplossingen: We gaan niet iedere keer hetzelfde wiel opnieuw uitvinden.
  \item standaardisatie: Design patterns vormen een mooie kapstok van begrippen waar een team van specialisten zich aan kan vasthouden. 
  \end{itemize}
\end{frame}

\begin{frame} \frametitle{Andere redenenen}
  \begin{itemize}
  \item het in perspectief plaatsen van denkwijzes
  \item bepalen of een gevonden oplossing de \emph{juiste} is i.p.v. slechts \emph{een} oplossing die werkt
  \item tot code komen die gemakkelijker te veranderen is
  \item verhoging van flexibiliteit, niets zo erg als specificaties die tijdens het implementeren veranderen\ldots komt echter vaak voor
  \item gemakkelijker alternatieven vinden
  \end{itemize}
\end{frame}

\section{Facade}

\begin{frame} \frametitle{Het facade pattern}
  \begin{itemize}
  \item naam: het facade patroon
  \item doel: een complex systeem kunnen gebruiken zonder er alles van af te hoeven weten
  \item hoe: bouw een interface voor je client applicatie
  \item gevolgen: beperkte functionaliteit
  \end{itemize}
\end{frame}

\begin{frame}[fragile]\frametitle{Voorbeeld}
Stel je een een ingewikkelde database applicatie voor, waarin wordt bijgehouden:
\begin{itemize}
\item wie waar woont
\item wat een huis kost
\item huur of koop
\item vrijstaand huis/rijtjeshuis/flat
\item tuin of niet
\item enz.
\end{itemize}

\end{frame}


\begin{frame} \frametitle{Voorbeeld}
  De opdracht:
  \begin{itemize}
  \item bouw een programma, waarmee de klant kan zien of het aantal koopwoningen in een bepaalde stad groeit of slinkt t.o.v. het aantal huurwoningen
  \end{itemize}
Traditionele oplossing:
\begin{itemize}
\item bouw een programma dat helemaal integreert in het bestaande programma\ldots
  \begin{itemize}
  \item zeer ingewikkeld
  \item foutgevoelig
  \item levert een hele hoop onnodige rompslomp op waar de klant niet op zit te wachten
  \end{itemize}
\end{itemize}
\end{frame}


\begin{frame} \frametitle{De facade}
  \begin{itemize}
\item de client applicatie hoeft niet alles van het ingewikkelde systeem te weten
\item bouw een een facade klasse die met (onderdelen van) het complexe systeem communiceert
\item de facade vormt een interface voor de client applicatie:
  \begin{itemize}
  \item simpeler
  \item gericht op wat de client applicatie nodig heeft zonder overbodige rommel
  \item respecteert de compliteit van het systeem zonder deze te willen wijzigen
  \end{itemize}
\end{itemize}
\end{frame}


\section{Adapter}


\begin{frame}[fragile]\frametitle{Het adapter pattern}
  \begin{itemize}
  \item naam: het adapter patroon
  \item doel: de client applicatie laten samenwerken met een ingewikkeld systeem zonder dit systeem te hoeven aanpassen
  \item hoe: bouw een adapter tussen systeem en client applicatie, de adapter is een soort vertaler
  \item gevolgen: bestaande systemen kunnen worden uitgebreid zonder rekening te hoeven houden met de interface
  \end{itemize}
\end{frame}


\begin{frame} \frametitle{Verschillen facade v.s. adapter}
Facade:
\begin{itemize}
\item bestaan er reeds classes? Ja
\item bestaat er reeds een interface waar we ons aan moeten houden? Nee (die maak je zelf)
\item is een eenvoudigere interface gewenst? Ja
\end{itemize}
Adapter:
\begin{itemize}
\item bestaan er reeds classes? Ja
\item bestaat er reeds een interface waar we ons aan moeten houden? Ja
\item is een eenvoudigere interface gewenst? Nee
\end{itemize}
\end{frame}


\begin{frame}[fragile]\frametitle{Voorbeeld}
Blender, een 3d modelleringsprogramma\ldots

Werking:
\begin{itemize}
\item bouw een 3d model
\item smeer bitmaps tegen alle vlakken
\item render het model
\end{itemize}
\end{frame}



\begin{frame} \frametitle{Blender: het probleem}
  Een groep slimmeriken is bezig met het bouwen van een nieuwe rendering engine\ldots
  \begin{itemize}
  \item de engine bestaat uit een enorme hoeveelheid classes met daarin vele methodes
  \item men wil blender met deze engine laten werken
  \item de technici werken vrolijk verder aan hun engine en vragen de heren van Blender niets
  \end{itemize}
Probleem: laat Blender met deze nieuwe engine werken
\end{frame}

\begin{frame} \frametitle{Blender: oplossing}
  Bouw een adapter:
  \begin{itemize}
  \item de adapter klasse heeft een pointer naar het rendering object in zich.
  \item methodes die vanuit Blender worden aangeroepen, worden door de adapter vertaald naar methodes die de engine begrijpt
  \end{itemize}
\end{frame}




\end{document}
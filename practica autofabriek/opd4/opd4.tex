\documentclass[fleqn,11pt]{article}

\usepackage[dutch]{babel}
\usepackage{epsfig}

\title{Design patterns opdracht 4: de autofabriek}
\author{W. Oele}

\begin{document}
\sffamily
\maketitle
\newpage
\emph{Lees eerst de hele opdracht alvorens te beginnen.}

\subsection*{Inleiding}
We hebben inmiddels twee soorten onderdelen ontworpen:
\begin{itemize}
\item brandstoftanks
\item motoren
\end{itemize}

We gaan in deze opdracht auto' s bouwen, d.w.z.:
\begin{itemize}
\item We combineren eerst de onderdelen tot een werkende auto.
\item We ontwerpen een fabriek die verschillende soorten auto' s kan produceren.
\end{itemize}

\subsection*{Opdracht 1}
Ontwerp een auto met daarin:
\begin{itemize}
\item een motor
\item een brandsfoftank
\end{itemize}
Geef de auto wat extra eigenschappen, bijvoorbeeld een merk en een type. Maak deze zichtbaar in de g.u.i.


\subsection*{Opdracht 2}
Ontwerp een fabriek die uit de eerder ontworpen onderdelen een auto produceert. Voorwaarden:
\begin{itemize}
\item De fabriek heeft zijn eigen g.u.i.
\item De gebruiker moet via de g.u.i. kunnen bepalen wat voor soort auto er geproduceerd wordt.
\item Een geproduceerde auto moet ``getest'' kunnen worden door de gebruiker (een ``proefritje'' maken via de g.u.i.).
\end{itemize}

\subsection*{Opdracht 3}
Programmeer je programma.
\begin{center}
\fbox{Begin pas met programmeren als de docent je ontwerp in u.m.l. heeft goedgekeurd! }  
\end{center}

\end{document}
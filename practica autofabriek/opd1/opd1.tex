\documentclass[fleqn,11pt]{article}

\usepackage[dutch]{babel}
\usepackage{epsfig}
\usepackage{hyperref}


\title{Design patterns opdracht 1}
\author{W. Oele}

\begin{document}
\sffamily
\maketitle
\newpage
\subsection*{Opdracht 1}
Vorig jaar maakte je een practicumopdracht, waarin je gevraagd werd het simuleren van 1 miljoen vluchten met een vliegtuig te programmeren. Zie: \href{http://med.hro.nl/oelew/infprg02/opd1.pdf}{http://med.hro.nl/oelew/infprg02/opd1.pdf}\\ 

Maak een klassendiagram in u.m.l. van het betreffende vraagstuk. Hanteer hierbij de notatiewijze uit het boek. Het maken van een interactiediagram is \emph{niet} nodig.

\subsubsection*{Aanwijzingen}
Een klassendiagram dient de essentie van een ontwerp weer te geven. Onbelangrijke details dienen derhalve te worden weggelaten. 
\subsection*{Opdracht 2}
In de klassen \verb|Pilot, Flap| en \verb|Engine|  vliegtuig vind je de \verb|calculate| methode. Het feit dat deze methode in elk van deze klassen voorkomt, duidt op een slecht ontwerp.  

Verbeter je ontwerp uit opdracht 1, opdat methodes zoveel mogelijk op slechts \'e\'en plek in het programma voorkomen. 

% \subsection*{Opdracht 3}
% \subsection*{Opdracht 4}

\end{document}
\documentclass[fleqn,11pt]{article}

\usepackage[dutch]{babel}
\usepackage{epsfig}

\title{Design patterns opdracht 3: motor en tank}
\author{W. Oele}

\begin{document}
\sffamily
\maketitle
\newpage
\emph{Lees eerst de hele opdracht alvorens te beginnen.}

\subsection*{Inleiding}
De fabrikant waar we de auto voor bouwen, wenst verschillende typen auto' s te kunnen produceren:
\begin{itemize}
\item een op benzine rijdende auto
\item een op gas rijdende auto
\item een op diesel rijdende auto.
\end{itemize}

\subsection*{Opdracht 1}
Verwerk een brandstoftank in het ontwerp van practicumopdracht 1. Voorwaarden:
\begin{itemize}
\item De tank heeft een bepaalde inhoud.
\item De motor moet realtime brandstof uit de tank kunnen halen.
\item De motor dient af te slaan als de tank leeg is.
\item De hoeveelheid aanwezige brandstof moet op het dashboard af te lezen zijn.
\item In de g.u.i. moet duidelijk worden wat voor soort brandstof de auto gebruikt.
\end{itemize}

\subsection*{Opdracht 2}
Het werken met verschillende soorten brandstof heeft nogal wat gevolgen voor het type motor en de hoeveelheid brandstof die per tijdseenheid wordt verbruikt. Verwerk in je ontwerp de mogelijkheid verschillende typen motoren in te bouwen. 

\subsection*{Opdracht 3}
Programmeer je programma en maak enkele brandstoftank/motor combinaties ter demonstratie. 

\begin{center}
\fbox{Begin pas met programmeren als de docent je ontwerp in u.m.l. heeft goedgekeurd! }  
\end{center}


\end{document}